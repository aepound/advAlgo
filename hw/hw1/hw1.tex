\documentclass{article}
%\standaloneconfig{class=book}

%%%
% If thin isn't
%
%%%
\makeatletter
\def\@getcl@ss#1.cls#2\relax{\def\@currentclass{#1}}
\def\@getclass{\expandafter\@getcl@ss\@filelist\relax}
\@getclass
\typeout{This document uses \@currentclass\space class...}
\@ifclassloaded{usuthesis}% {class}{<true>}{<false>}
  {}% Don't want to use this, if we're in usuthesis class. 
  { % If we are NOT being included in the thesis or beamer,
    % then use these package...
    \@ifclassloaded{beamer}
    {}% Don't want to use this in the beamer class, either. 
    { %
      \usepackage{fullpage}
      \usepackage{setspace}
      \doublespacing
    }
  }% 

\usepackage{graphicx}
\input{symbolmac}
\usepackage{amsmath}
\usepackage{amssymb}
\usepackage{euscript}
\usepackage{latexsym}
\makeatletter
    \@ifclassloaded{beamer}
    {}% Don't want to use this in the beamer class, either. 
    { %
\usepackage[dvipsnames,usenames]{xcolor}
}
\usepackage{tikz}
\usetikzlibrary{calc}

\makeatother
\usepackage{subcaption}
\usepackage{booktabs}


% I thought about adding this in to be able to switch between lscape
% and pdflscape... but decided I didn't need to...
%\usepackage{ifpdf}
%\ifpdf
%\usepackage{pdflscape}
%\else
\usepackage{lscape}
%\fi

%\usepackage{cleveref}

\providecommand{\abs}[1]{\lvert#1\rvert}
\providecommand{\norm}[1]{\lVert#1\rVert}
\definecolor{dark-gray}{gray}{0.2}
\definecolor{light-gray}{gray}{0.3}
\newcommand{\myrule}{\textcolor{light-gray}{\noindent\rule{\linewidth}{0.07mm}}\\}
\newcommand{\editmark}{\begin{center}\textcolor{light-gray}{\noindent\rule{\linewidth}{0.07mm}}\\ 
    To be Edited:\end{center}} 
\renewcommand{\matlab}{{\sc Matlab }} 
\newcommand{\edit}[1]{\editmark \textcolor{dark-gray}{#1}\\\myrule}
%\newcommand{\comment}[1]{}
\newcommand{\nochapbibs}[1]{#1}
\newcommand{\etal}{et~al.~}

%\usepackage{comment}




\usepackage{comment}
\usepackage{enumitem}
\usepackage{algpseudocode}

\title{CS 5050: Homework 1}             
\author{Andrew Pound}
\date{\today}

\begin{document}
\maketitle

I worked on this assignement with Chad Cummings.
\section{}
\begin{enumerate}[label=(\alph*)]
\item  Suppose we are comparing implementations of insertion sort and 
merge sort on the same machine. For inputs of size n, insertion sort
runs in $8n^2$ steps, while merge sort runs in $64n \log n$ steps. For
which values of $n$ does insertion sort beats merge sort? {\bf (5
  points)} 
\item[]
There are a few different ways to do this. The first way that we
attacked it was that we solved the inequality equation to easiest way
to solve this is to tabulate the answers for each side of the equatoin
and find the point that they cross.  (This would actually get a bit
long for this particular problem).  So instead a plot of the two
functions can be used to see where it crosses.

Equations:
\begin{equation}\label{eq:1a}
  \begin{split}
    8n^2 &\le 64n\log n\\
    \frac{1}{8} &\le  \frac{1}{n}\log n\\
    2^{\frac{1}{8}} &\le n^{\frac{1}{n}}
  \end{split}
\end{equation}

Plotting both sides of this, produces the graph below.
\begin{figure}[h!t]
  \centering
  \includegraphics[width=.7\linewidth]{prob1a}
  \caption{Plot of the inequality in equation \ref{eq:1a}.}
  \label{fig:prob1a}
\end{figure}

From this we can see that the cross over point is around 43.
Tabulating the values of $n^{1/n}$ in Table \ref{tab:1a}, the
neighborhood of 43 and 
\begin{table}[h!t]
  \centering
  \caption{Tabulation of values around $n=43$.}
  \label{tab:1a}
  \begin{tabular}{cc}
    $n$ & $n^{1/n}$ \\
    40	 & 1.09661 \\
41	 & 1.0948 \\
42	 & 1.09307 \\
43	 & 1.09141 \\
44	 & 1.08981 \\
45	 & 1.08827 \\

  \end{tabular}
\end{table}
comparing them with the value of $2^{1/8} = 1.0905$, we can see that
any number greater than 43 satisfies the inequality.  


\item  What is the smallest value of n such that an algorithm whose
running time is $100n^2$ runs faster than an algorithm whose running
time is $2^n$ on the same machine? {\bf (5 points) }
\item[] This can be solved in much the same manner as the previous
  problem. First, Let's examine the inequality.
  \begin{equation}\label{eq:1b}
    \begin{split}
      100 n^2 & \le 2^n\\
      \log_2(100n^2) & \le log_2(2^n) = n\\
      \frac{2}{n} \log(10n) &\le 1\\
      \frac{1}{n} \log(10n) &\le \frac1 2\\
      \log\left(n^{\frac{1}{n}}\right) &\le \frac{1}{2}\\
      \left(10n\right)^{\frac{1}{n}} &\le \sqrt{2}
    \end{split}
  \end{equation}
\end{enumerate} 

The plot of this function is seen in Figure \ref{fig:1b}, and the
tabulation around the cross over is in Table \ref{tab:1b}.  This time
we ar comparing with the value $\sqrt{2} = 1.4142$, and we can see
that the cross over happens at $n=15$.
\begin{figure}[h!t]
  \centering
  \includegraphics[width=.7\linewidth]{prob1b}
  \caption{Plot of the inequality in equation \ref{eq:1b}.}
  \label{fig:1b}
\end{figure}

\begin{table}[h!t]
  \centering
  \caption{Tabulation of values around $n=15$.}
  \label{tab:1b}
  \begin{tabular}[h]{cc}
    $n$ & $(10n)^{1/n}$ \\
    12	 & 1.23008 \\
13	 & 1.21811 \\
14	 & 1.20744 \\
15	 & 1.19786 \\
16	 & 1.18921 \\
17	 & 1.18135 \\

  \end{tabular}
\end{table}

\vspace{3em}
\section{}
We have discussed the insertion sort algorithm in class and we showed
that in the worst case the algorithm runs in $\Theta(n^2)$ time. You
may wonder whether the algorithm can run asymptotically faster for
certain ``average'' cases. In this exercise, we will examine the running
time of the insertion sort algorithm on some average cases. Let $A$ be 
the input array of n elements.

Note: For each of the questions, please give your answers in the
big-Theta notation and also briefly explain how you obtain your answer.

\begin{enumerate}[label=(\alph*)]
\item Consider the case where the first $n/2$ elements of $A$ have
already been sorted \emph{increasingly} and the last $n/2$ elements
have been sorted \emph{decreasingly}. For example, $A = \{1, 3, 5, 7,
8, 6, 4, 2\}$. Suppose we want to sort all elements of $A$ in
increasing order by using the insertion sort algorithm. What would be
the running time of the algorithm? What if the first $n/2$ elements of
$A$ have already been sorted decreasingly and the last $n/2$ elements
have been sorted increasingly? For example, $A = \{8, 6, 4, 2, 1, 3,
5, 7\}$. What would be the running time in this case? {\bf (10 points)}

\item[] The insertion sort algorithm that we discussed in class is
  given as
  \begin{algorithmic}
    \For{$i=2\dots n$}
    \State $x\gets A[i]$
    \State $j \gets i-1$
    \While{$A[j-1] < x \And j \ge 1$}
    \State $A[j+1]\gets A[j]$
    \State $j \gets j-1$
    \EndWhile
    \State $A[j+1]\gets x$
    \EndFor
  \end{algorithmic}
The outer for loop consists of $4c$ operations, and the inner loop
(when run) consists of $2c$ operations.
Now suppose that  the elements of our array are ordered in such a  way
that the first half of the array are already in ascending order, and
the second half is the worst case (descending order) and need to be
merged into the first half.  If that is the case, then the first half
of the algorithm is a best case scenario, and only uses
$4c\frac{n}{2}$ operations.  The second half, though will utilize the
inner for-loop to move some of the elements in the first half of the
array in order to insert in the correct place. 
By considering the first elements in the second half, we can determine
a relation between the number of times in the inner loop and the
iteration index.  If the array is in the form specified, then the
first element of the second half will not require any moving to place
it.  The second will require 2 moves (the first element of the 2nd
half and the last element of the first half).  The third element in
the 2nd half will require 4 moves.  We see that for $i > \frac{n}{2}$,
the number of times in the inner loop will be $2(i-\frac{n}{2})$.
Thus, we can stick this into an expression for the total number of
operations
\begin{equation}
  \underbrace{4cn}_{\text{outer loop}} + \underbrace{\sum_{i =
    \frac{n}{2}}^n2\left(i-\frac{n}{2}\right)\left(2c\right).
}_{\text{inner loop}} 
\end{equation}
Now expanding this, we get a total of 
\begin{equation}
\begin{split}
  (4cn) + 4c\sum_{i=0}^{n/2}i &= 4cn +
  4c\frac{\frac{n}{2}\left(\frac{n}{2} + 1\right)}{2}\\
  &= 4cn + 2c\left(\frac{1}{4}n^2 + \frac{1}{2}n\right)\\
  &= 4cn + \frac{1}{2}n^2c + cn\\
  &= \frac{1}{2}cn^2 + 5cn
  &= O(n^2}
\end{split}
\end{equation}

%The second ordering 



\item Consider the case where the elements of $A$ with odd indices have
already been sorted increasingly and elements with even indices have
been sorted decreasingly (we assume the index of $A$ starts from
1). For example, $A = \{1, 8, 2, 7, 3, 6, 4, 5\}$. Suppose we want to
sort all elements of $A$ in increasing order by using the insertion
sort algorithm. What would be the running time of the algorithm?

What if the elements of $A$ with odd indices have already been sorted
\emph{decreasingly} and elements with even indices have been sorted 
\emph{increasingly}. For example, $A = \{8, 1, 7, 2, 6, 3, 5,
4\}$. What would be the running time in this case?  {\bf (10 points)}

\item[] 
Ok, we'll begin by looking at how the algorithm works on the example
sequence.  The results of the steps of the algorithm are shown below.
The bolded numbers identify values that needed to be moved (i.e. the
values for which the inner loop was used). The first line is the
original sequence.
\begin{equation}
  \begin{matrix}
    1 & 8 & 2 & 7& 3  &6 & 4 & 5\\
    1 &&&&&&&\\
    1 & 8 &&&&&&\\
    1 & 2 & {\bf 8}&&&&&\\ 
    1 & 2 & 7 & {\bf 8}&&&&\\
    1 & 2 & 3 & {\bf 7}&{\bf 8}  &&&\\
    1 & 2 & 3 & 6& {\bf 7}&{\bf 8}  &&\\
    1 & 2 & 3 & 4& {\bf 6}& {\bf 7}&{\bf 8}  &\\
    1 & 2 & 3 & 4& 5  &{\bf 6} & {\bf 7}&{\bf 8} \\
  \end{matrix}
\end{equation}
From this we can see that for each pair of iterations, the number of
moves ends up being predictable.  In order to count the time, we then
start with the equation
\begin{equation}
  \underbrace{4cn}_{\text{outer loop}} + \underbrace{2 \sum_{i=1}^{n/2}
    (i-1)(2c)}_{\text{inner loop}}.
\end{equation}
The outer loop has $4c$ operations and happens for each element in the
array, thus we get $4cn$.   Each pair of iterations is hitting the
inner loop for the same number of times, namely $i-1$, for the $i$th
pair. The inner loop consists of $2c$ operations, thus we get the
summation over all the pairs, and multiply by 2 to get the full effect
of the inner loop.  This is
\begin{equation}
  \begin{split}
    2 \sum_{i=1}^{n/2}
    (i-1)(2c) &= 4c \sum_{i=1}^{n/2}(i - 1)\\
    &= 4c \sum_{i=1}^{n/2}i  - 4c\frac{n}{2}\\
    &= 4c \frac{\frac{n}{2}\left(\frac{n}{2} + 1\right)}{2} -
    4c\frac{n}{2}\\ 
    &= 2c\left(\frac{n^2}{4} + \frac{n}{2} - n\right)\\
    &= 2c\left(\frac{1}{4}n^2 - \frac{1}{2}n\right)
  \end{split}
\end{equation}
The entire operations count together then would be
\begin{equation}
  \begin{split}
    4cn + 2c\left(\frac{1}{4}n^2 - \frac{1}{2}n\right)&= 
    4cn + \frac{1}{2}cn^2 - cn\\
    &= \frac{1}{2}cn^2 + 3cn
    &= O(n^2}
  \end{split}
\end{equation}
Now for the other example, let's see what it looks like:
\begin{equation}
  \begin{matrix}
    8& 1& 7& 2& 6& 3& 5 &4 \\
    8 &&&&&&&\\
    1 & {\bf 8} &&&&&&\\
    1 & 7 & {\bf 8}&&&&&\\ 
    1 & 2 & {\bf 7} & {\bf 8}&&&&\\
    1 & 2 & 6 & {\bf 7}&{\bf 8}  &&&\\
    1 & 2 & 3 & {\bf 6}& {\bf 7}&{\bf 8}  &&\\
    1 & 2 & 3 & 5& {\bf 6}& {\bf 7}&{\bf 8}  &\\
    1 & 2 & 3 & 4& {\bf 5}  &{\bf 6} & {\bf 7}&{\bf 8} \\
  \end{matrix}
\end{equation}
Essentially, the idea is the same, in that the iterations can be split
into pairs such that a pair will use the inner loop (to move larger
numbers) the same number of times.  Thus the analysis is (almost)
exactly the same: $ O(n^2}$.


\end{enumerate}


\section{}
For each of the following pairs of functions, indicate whether it is
one of the three cases: $f (n) = O(g(n))$, $f (n) = \Omega(g(n))$, or

$f (n) = \Theta(g(n))$. (30 points)
\begin{enumerate}[label=(\alph*)]
\item $f (n) = 100n + \log n$ and $g(n) = 6n + \log^2 n$.
\item[] $f(n) = \Omega(g(n))$.  My reasoning is this:
  \begin{equation}
    \begin{split}
      f(n) &\: ?\: g(n)\\
      100n + \log n &\: ? \: 6n+]log^2 n\\
      94n &\: ?\: \log^2 n - \log n 
    \end{split}
  \end{equation}
Now, $\log^2 n - \log n < \log^2 n$, and comparing this as above, we
see that 
\begin{equation}
  \begin{split}
    94n > \log^2 n > \log^2 n -\log n.
  \end{split}
\end{equation}
Thus, we can say that $f(n) = \Omega(g(n))$.
\item $f (n) = 20 \log n + 4$ and $g(n) = \log n^2 - 100$.
\item[] $f(n) = \Theta(g(n))$.  Playing with the equations a bit, we
  can see
  \begin{equation}
    \begin{split}
      f(n) & \: ? \: g(n)\\
      20\log n + 4 &\: ? \: \log n^2 - 100\\
      20 \log n + 4 &\: ? \: 2\log n - 100.
    \end{split}
  \end{equation}
Now, if we choose $c_1 = 20$ and $c_2 = 10$, then we get an asymptotic
upper and lower bound, respectively, by forming the equations $c_i
g(n)$.  Thus, we know that $f(n) = \Theta(g(n))$.
\item $f (n) =\frac{n^2}{\log n}$ and $g(n) = n \log^2 n$.
\item[] $f(n) = \Omega(g(n))$. Let's play with the equations:
  \begin{equation}
    \begin{split}
      \frac{n^2}{\log n} &\: ? \: n \log^2 n\\
      n &\: ? \: \log^3 n.
    \end{split}
  \end{equation}
Then using a nifty formula:
  \begin{equation}
    \log^k n = O(n^d) \quad \forall\: k>0, \: d > 0,
  \end{equation}
we can see that $d = 1$, and $k = 3$, and thus, $g(n) = O(f(n))$. 
Or in other words, $f(n) = \Omega(g(n))$.
\item $f (n) =\sqrt{n}$ and $g(n) = \log^5 n$.
\item[] $f(n) = \Omega(f(n))$. Using the same nifty equation from
  above, we can see that $g(n) = O(f(n))$, as $k = 5$, and $d = 1/2$.
  Thus flipping this around, we see that $f(n) = \Omega(f(n))$.
\item $f (n) = n2^n$ and $g(n) = 3^n$ .
\item[] $f(n) = O(g(n))$.  We will make use of our other nifty
  formula 
  \begin{equation}
    n^d = o(e^n), \quad \forall\: e > 1.
  \end{equation}
First, let's play a bit with our equation
\begin{equation}
\begin{split}
  n2^n &\: ?\: 3^n\\
  n &\: ? \:  \left(\frac{3}{2}\right)^n.
\end{split}
\end{equation}
This helps us to see that when $d = 1$ and $ e = 3/2$, then 
we know that $f(n) = o(g(n))$ and therfore, $f(n) = O(g(n))$.
\item $f (n) = 4n \log n$ and $g(n) = n \log_3 n$.
\item[] $f(n) = \Theta(g(n))$. Playing with the equation, we get
  \begin{equation}
    \begin{split}
    4n \log n &\: ? \: n \log_3 n\\
    4 \log n &\: ? \: \frac{\log n}{\log 3}\\
    4 \log n &\: ? \: \frac{1}{\log 3} \log n\\
    4 &\: ? \: \frac{1}{\log 3}
    \end{split}
  \end{equation}
Now we see that $f(n) = \Theta(g(n))$.
\end{enumerate}


Note: For each question, you only need to give your answer and the
proof is not required. 


\section{}
The \emph{knapsack problem} is defined as follows: Given as input a
knapsack of size $K$ and $n$ items whose sizes are $k_1 , k_2 ,
\dots , kn$ , where $K$ and $k_1 , k_2 , \dots , k_n$ are all real
numbers, find a full ``packing'' of the knapsack (i.e., choose a
subset of the n items such that the total sum of the sizes of the
items in the chosen subset is exactly $K$). 

It is well known that the knapsack problem is NP-complete, which
implies that it is very likely that efficient algorithms (i.e., those
with a polynomial running time) for this problem do not exist. Thus,
people tend to look for good {\bf approximation algorithms} for
solving this problem. In this exercise, we relax the constraint of the
knapsack problem as follows. We still seek a packing of the knapsack,
but we need not look for a ``full'' packing of the knapsack; instead,
we look for a packing of the knapsack (i.e., a subset of the n input
items) such that the total sum of the sizes of the items in the chosen
subset is at least $K/2$ (but no more than $K$). This is called a
\emph{factor of 2 approximation solution} for the knapsack problem. To 
simplify the problem, we assume that a factor of 2 approximation
solution for the knapsack problem always exists, i.e, there always
exists a subset of items whose total size is at least $K/2$ and at
most $K$. 

Design a polynomial time algorithm for computing a factor of 2
approximation solution for this problem, and analyze the running time
of your algorithm (in the big-O notation). If your algorithm runs in
$O(n)$ time and is correct, then you get {\bf 5 extra points}. {\bf(20
  points)} 

Note: You are required to clearly describe the main idea of your
algorithm. Although the pseudo-code is not required, you may also give
the pseudo-code if you feel it is helpful for you to describe your
algorithm. (The reason I want to see the algorithm description instead
of only the code or pseudo-code is that it would be difficult to
understand another person's code without any explanation.) You also
need to briefly explain why your algorithm works, i.e., why your
algorithm can produce a factor of 2 approximation solution. Finally,
please analyze the running time of your algorithm. 

\paragraph{My algorithm}





In order to find a factor of 2 approximation, we will make use of the
assumptions that there definitely exists such an approximation, and
that all of the $k_i$ sizes are positive.

The algorithm is as follows:

Iterate through the list by pairs, finding the largest of each pair
and moving it to the lower position in the pair.
Then iterate and add the largest of each pair (unless it is too large
(i.e. makes the knapsack over full).
Then, if at the end of this loop, that knapsack is not at least 1/2
full, then iterate through the smaller of the pairs, adding them,
unless they overfill the knapsack.

Because there exists a factor of 2 approximation, this algorithm will
terminate when either the solution is found or the entire list is
added during the 3rd loop, which will then constitute a solution
(because one has to exist).

\end{document}