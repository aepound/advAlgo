\documentclass{article}
%\standaloneconfig{class=book}

%%%
% If thin isn't
%
%%%
\makeatletter
\def\@getcl@ss#1.cls#2\relax{\def\@currentclass{#1}}
\def\@getclass{\expandafter\@getcl@ss\@filelist\relax}
\@getclass
\typeout{This document uses \@currentclass\space class...}
\@ifclassloaded{usuthesis}% {class}{<true>}{<false>}
  {}% Don't want to use this, if we're in usuthesis class. 
  { % If we are NOT being included in the thesis or beamer,
    % then use these package...
    \@ifclassloaded{beamer}
    {}% Don't want to use this in the beamer class, either. 
    { %
      \usepackage{fullpage}
      \usepackage{setspace}
      \doublespacing
    }
  }% 

\usepackage{graphicx}
\input{symbolmac}
\usepackage{amsmath}
\usepackage{amssymb}
\usepackage{euscript}
\usepackage{latexsym}
\makeatletter
    \@ifclassloaded{beamer}
    {}% Don't want to use this in the beamer class, either. 
    { %
\usepackage[dvipsnames,usenames]{xcolor}
}
\usepackage{tikz}
\usetikzlibrary{calc}

\makeatother
\usepackage{subcaption}
\usepackage{booktabs}


% I thought about adding this in to be able to switch between lscape
% and pdflscape... but decided I didn't need to...
%\usepackage{ifpdf}
%\ifpdf
%\usepackage{pdflscape}
%\else
\usepackage{lscape}
%\fi

%\usepackage{cleveref}

\providecommand{\abs}[1]{\lvert#1\rvert}
\providecommand{\norm}[1]{\lVert#1\rVert}
\definecolor{dark-gray}{gray}{0.2}
\definecolor{light-gray}{gray}{0.3}
\newcommand{\myrule}{\textcolor{light-gray}{\noindent\rule{\linewidth}{0.07mm}}\\}
\newcommand{\editmark}{\begin{center}\textcolor{light-gray}{\noindent\rule{\linewidth}{0.07mm}}\\ 
    To be Edited:\end{center}} 
\renewcommand{\matlab}{{\sc Matlab }} 
\newcommand{\edit}[1]{\editmark \textcolor{dark-gray}{#1}\\\myrule}
%\newcommand{\comment}[1]{}
\newcommand{\nochapbibs}[1]{#1}
\newcommand{\etal}{et~al.~}

%\usepackage{comment}




\usepackage{comment}
\usepackage{enumitem}
\usepackage{algpseudocode}
\setcounter{MaxMatrixCols}{20}
\usepackage{algpseudocode}

\usepackage{mathtools}
\DeclarePairedDelimiter{\ceil}{\lceil}{\rceil}
\DeclarePairedDelimiter{\floor}{\lfloor}{\rfloor}

\title{CS 5050: Homework 3}
\author{Andrew Pound}
\date{\today}


\begin{document}
\maketitle

%\begin{enumerate}[label=(\alph*)]
I worked on this assignment with Chad Cummings.
\section{}
 In the SELECTION algorithm we studied in class, the input numbers are
 divided into groups of five. Will the algorithm still work in linear
 time if they are divided into groups of seven? 
Please justify your answer.


\paragraph{Answer}
Yes. The recurrence for the complexity of the selection algorithm
would now be
\begin{equation*}
  T(n) = T\left(\max\left\{\left|A_1\right|,
      \left|A_2\right|\right\}\right)  + T\left(\frac{n}{7}\right) 
      + n. 
\end{equation*}
Now we need to calculate the $\max\left\{\left|A_1\right|,
      \left|A_2\right|\right\}$.
Using the same reasoning as in class, we can draw the diagram below.
Break the numbers into groups of 7, and sort them in ascending order. 
\newcommand{\yit}{\mathit{x}}
\begin{equation*}
  \text{7 rows}
  \underbrace{
  \begin{bmatrix}
    \xbf &  &\xbf &  &\cdots &  &\xbf & \xbf & 
     &  &  &  &  &  & \\
%     \vee&&\vee&&&&\vee&\vee&
%     &&&&&&\\
    \xbf &  &\xbf &  &\cdots &  &\xbf & \xbf & 
     &  &  &  &  &  & \\
%     \vee&&\vee&&&&\vee&\vee&
%     &&&&&&\\
    \xbf &  &\xbf &  &\cdots &  &\xbf & \xbf & 
    &  &  &  &  &  & \\
     \vee&&\vee&&&&\vee&\vee&
     &&&&&&\\
    \xbf & < &\xbf & < &\cdots & < &\xbf & y & 
    \yit & < &\cdots & < & \yit & < & \yit\\
     &&&&&&&\vee&
     \vee&&&&\vee&&\vee\\    
        &  &  &  &  &  & & \yit &
    \yit &  &\cdots &  & \yit &  & \yit\\
        &  &  &  &  &  & & \yit&
    \yit &  &\cdots &  & \yit &  & \yit\\
        &  &  &  &  &  & & \yit&
    \yit &  &\cdots &  & \yit &  & \yit
  \end{bmatrix}}_{n/7 \text{ columns}}
\end{equation*}
We have to calculate the number of the bold x's. 
\begin{equation*}
  N_b = \frac{4}{7}\frac{n}{2} - 1 = \frac{4}{14}n - 1.
\end{equation*}
$N_b$ is the number of items that are less than $y$ for sure. Thus we
can say that at least $N_b = \frac{4}{14}n < y$. So we know that
$n-N_b = \frac{10}{14}n = \frac{5}{7}n$ is the maximum number that could
be larger than $y$. Thus the maximum size for $\left| A_2\right|$ is
then $\frac{5}{7}n$.
The analysis is similar for the maximum value $\left|A_1\right|$.
Thus, the recurrence is 
\begin{equation*}
  T(n) = T\left(\frac{5}{7}n\right) + T\left(\frac{1}{7}n\right) + n.
\end{equation*}
This can be solved to give linear time. So, we will try this by
substitution.
We want to find $c$ such that 
\begin{equation*}
  T(n) \le cn.
\end{equation*}
Let's assume $T\left(\frac{5}{7}n\right) \le c\frac{5}{7}n,$ and
$T\left(\frac{1}{7}n\right) \le c\frac{1}{7}n.$  Then we can consider
\begin{equation*}
  T(n) \le c\frac{5}{7}n + c\frac{1}{7}n + n  = c\frac{6}{7}n + n
  \stackrel{?}{\le} cn. 
\end{equation*}
Moving things areound, we can see that 
\begin{equation*}
  n \stackrel{?}{\le} \frac{c}{7}n,
\end{equation*}
which is true if $c \ge 7$ (with $n_0 = 1$).  Thus we can say that 
$T(n) = O(n)$.



\section{}
Suppose you are consulting for an oil company, which is planning a
large pipeline (called the main pipeline) running horizontally from
east to west through an oil field of n wells. From each well, a spur
pipeline is to be connected directly to the main pipeline along a
shortest path (going to either the north or the south), as shown in
Figure 1. 

Suppose there are $n$ wells, represented by $n$ points $p_1, p_2,
\dots , p_n$ in the plane. We are given the x- and y-coordinates of the
$n$ wells $p_i = (x_i , y_i)$ for $i = 1, 2, \dots, n$. Note that the
wells are not given in any sorted order. Our goal is to pick an
optimal location for the main pipeline (i.e., find the y-coordinate of
the main pipeline) such that the total sum of the lengths of the spur
pipelines is minimized. For simplicity, we assume no two wells have
the same x-coordinate or y-coordinate. Design an $O(n)$ time algorithm
to compute an optimal location for the main pipeline.

\paragraph{Answer}
The optimal placement for the line will be the median of the wells'
posistions in the y-direction.  If there are an even number of wells,
then anywhere in between the two medians will be sufficient to
minimize the amount of spur piping.  The reasoning is that if you are
anywhere outiside the median, then there are more spurs on the side
toward the median, than on the other side.  Thus, if you move the main
towards the median, then you will be shortening the amount of spur
piping, because of the differential of the number of spurs on either
side. This effect is countered, when the main reaches the median, as
any further shifting will add piping.

Now because we have now reduced this to finding the median of a set of
numbers (the y-coordinates), we can just use the Selection algorithm.
The Selection algorithm has linear time complexity, thus satisfying
our constraints, and can be made to return the median by using
{\sc Selection}$\left(A,1,n,\floor{n/2}\right).$


\section{}
We consider a \emph{weighted} version of Problem 2. Because of the
locations, 
the cost for building spur pipelines on different wells may be
different. For each well $p_i$, it is associated with a weight $w_i >
0$, representing the cost for building a spur line of a unit length,
and in other words, if the length of the spur line for $p_i$ is $l$,
then the cost for building the spur line is $w_i\cdot l$

For each well $p_i$, we are given its x- and y-coordinates $(x_i,
y_i)$ and weight $w_i$. Again, the wells are not given in any sorted
order. Our goal is to pick an optimal location for the main pipeline
(i.e., find the y-coordinate of the main pipeline) such that the total
sum of the cost of building the spur pipelines for all wells is
minimized. For simplicity, we assume no two wells have the same
x-coordinate or y-coordinate. Design an $O(n)$ time algorithm to
compute an optimal location for the main pipeline.  

\paragraph{Answer}
The best placing of the main pipe will be when the costs on either
side of the main line are equal.  The argument is the same as for the
previous problem, but utilizes the linear cost instead of just
length. 

We will need to use a weighted selection algorithm.  The idea is still
to find a ``median'' of the y-coordinates, but to also take into
account the costliness of the lengths of the spur lines.
So we will use the algorithm below.

{\singlespacing
\begin{algorithmic}
\Function{WeigtedSelection}{A,$C_1,C_2$}
  \State Find a pivot number $y$ in A \Comment{$O(n)$}
  \State $A_1 = \{a_i | a_i < y \}$    \Comment{$O(n)$}
  \State $A_2 = \{a_i | a_i > y \}$    \Comment{$O(n)$}
 

  \State Compute $W_1 = w(A_1)$   \Comment{$O(n)$}
  \State Compute $W_2 = w(A_2)$   \Comment{$O(n)$}
  %\State $i = |A_1|$ \Comment{$i$ is the size of $A_1$}
  %\State $j = |A_1|$ \Comment{$j$ is the size of $A_2$}
  %\If{ }
  %\EndIf
  \If{ $C_1 + W_1 = C_2 + W_2$}
    \State return y
  \ElsIf{$C_1 + W_1 > C_2 + W_2 $}
    \State return  \Call{WeigtedSelection}{$A_1,C_1,C_2+W_2$}% $A_1,y$ 
    \Comment{$O(|A_1|)$}
  \Else%{ $feas$}
    \State return \Call{WeigtedSelection}{$A_2,C_1+W_1,C_2$}
    \Comment{$O(|A_2|)$}
  \EndIf
\EndFunction
\end{algorithmic}
}

The pivot $y$ needs to be found as the median of the y-coordinates, as
in the regular Selection algorithm.  The algorithmic complexity is
given by
\begin{equation*}
     T(n) = T\left(\max\left\{\left|A_1\right|,
      \left|A_2\right|\right\}\right)  + T\left(\text{time to find
      pivot}\right)   + n. 
\end{equation*}
This is the same complexity as for the Selection algorithm, thus this
is a linear complexity algorithm.









\section{}
Suppose we are given a ``black-box'' which is a program procedure such
that given any real number $x$ as the input, the procedure can tell
whether $x$ is a feasible value in constant time. Further, we have the
following {\bf basic rules}: If $x$ is a feasible value, then any
number less 
than $x$ is also a feasible value; similarly, if $x$ is not a feasible
value, then any number larger than $x$ is not a feasible value. 

Given an array $A[1, \dots , n]$ of $n$ real numbers, we want to find
all feasible values of the elements of $A$ by using the black-box. For
each element $x$ of $A$, we can call the black-box on $x$ to determine
whether $x$ is feasible value. In this way, by calling the black-box 
$n$ times, we can find all feasible values of $A$ in $O(n)$
time. However, by making use of the above basic rules, it is possible
to find all feasible values of $A$ by calling the black-box
significantly less than $n$ times. For simplicity of discussion, we
assume no two numbers of $A$ are equal.


\begin{enumerate}[label=(\alph*)]
\item Design an $O(n\log n)$ time algorithm to find all feasible
  values of $A$ by calling the blackbox at most $O(\log n)$ times.
\item[]
A straightforward way to solve this problem is to sort the array
elements into ascending order.  This would require $O(n\log n)$
time. Then, once the elements have been sorted, then a binary search
could be done using the blackbox to find the largest element that
is feasible (or alternately, the smallest one that doesn't).  This
will prune the array, and will only require $\log n$ steps to find the
largest feasible element. Thus, the blackbox routine will only be
called $\log n$ times.  Thus the entire complexity of this will be
$T(n) = O(n\log n) + O(\log n) = O(n\log n).$

\item Improve your algorithm to $O(n)$ time such that the total number
  of calls on the blackbox is still at most $O(\log n)$.
\item[]
The idea here is we want to find the dividing line between the set of
feasible and not feasible set.  We can do this by using the Selection
algorithm and modifying it a bit.
The Algorithm can be given as


{\singlespacing
\begin{algorithmic}
\Function{FindFeas}{A}
  \State Find a pivot number $y$ in A \Comment{$O(n)$}
  \State $A_1 = \{a_i | a_i < y \}$    \Comment{$O(n)$}
  \State $A_2 = \{a_i | a_i > y \}$    \Comment{$O(n)$}
 
  \State $feas = $ \Call{BlackBox}{$y$}
  %\State $i = |A_1|$ \Comment{$i$ is the size of $A_1$}
  %\State $j = |A_1|$ \Comment{$j$ is the size of $A_2$}
  %\If{ }
  %\EndIf
  \If{ $feas$}
    \State return {\sc Concatenate}( $A_1,y$, \Call{FindFeas}{$A_2$} ) 
    \Comment{$O(|A_2|)$}
  \Else%{ $feas$}
    \State return \Call{FindFeas}{$A_1$}
    \Comment{$O(|A_1|)$}
  \EndIf
\EndFunction
\end{algorithmic}
}
The time complexity is given as
\begin{equation*}
     T(n) = T\left(\max\left\{\left|A_1\right|,
      \left|A_2\right|\right\}\right)  + T\left(\text{time to find
      pivot}\right)   + n. 
\end{equation*}
If we find the pivot $y$ as in the selection sort, then we can then
use the same analysis as that used to  find the complexity of the
Selection algorithm, and we can see that this will then have linear
time, while only using the Blackbox $\approx \log n$ times.



\end{enumerate}

Note that you are not responsible for designing the blackbox and just
assume it is given as a program procedure for you to use.
\end{document}