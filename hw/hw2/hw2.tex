\documentclass{article}
%\standaloneconfig{class=book}

%%%
% If thin isn't
%
%%%
\makeatletter
\def\@getcl@ss#1.cls#2\relax{\def\@currentclass{#1}}
\def\@getclass{\expandafter\@getcl@ss\@filelist\relax}
\@getclass
\typeout{This document uses \@currentclass\space class...}
\@ifclassloaded{usuthesis}% {class}{<true>}{<false>}
  {}% Don't want to use this, if we're in usuthesis class. 
  { % If we are NOT being included in the thesis or beamer,
    % then use these package...
    \@ifclassloaded{beamer}
    {}% Don't want to use this in the beamer class, either. 
    { %
      \usepackage{fullpage}
      \usepackage{setspace}
      \doublespacing
    }
  }% 

\usepackage{graphicx}
\input{symbolmac}
\usepackage{amsmath}
\usepackage{amssymb}
\usepackage{euscript}
\usepackage{latexsym}
\makeatletter
    \@ifclassloaded{beamer}
    {}% Don't want to use this in the beamer class, either. 
    { %
\usepackage[dvipsnames,usenames]{xcolor}
}
\usepackage{tikz}
\usetikzlibrary{calc}

\makeatother
\usepackage{subcaption}
\usepackage{booktabs}


% I thought about adding this in to be able to switch between lscape
% and pdflscape... but decided I didn't need to...
%\usepackage{ifpdf}
%\ifpdf
%\usepackage{pdflscape}
%\else
\usepackage{lscape}
%\fi

%\usepackage{cleveref}

\providecommand{\abs}[1]{\lvert#1\rvert}
\providecommand{\norm}[1]{\lVert#1\rVert}
\definecolor{dark-gray}{gray}{0.2}
\definecolor{light-gray}{gray}{0.3}
\newcommand{\myrule}{\textcolor{light-gray}{\noindent\rule{\linewidth}{0.07mm}}\\}
\newcommand{\editmark}{\begin{center}\textcolor{light-gray}{\noindent\rule{\linewidth}{0.07mm}}\\ 
    To be Edited:\end{center}} 
\renewcommand{\matlab}{{\sc Matlab }} 
\newcommand{\edit}[1]{\editmark \textcolor{dark-gray}{#1}\\\myrule}
%\newcommand{\comment}[1]{}
\newcommand{\nochapbibs}[1]{#1}
\newcommand{\etal}{et~al.~}

%\usepackage{comment}




\usepackage{comment}
\usepackage{enumitem}
\usepackage{algpseudocode}

\title{CS 5050: Homework 2}      
\author{Andrew Pound}
\date{\today}

\begin{document}
\maketitle

%\begin{enumerate}[label=(\alph*)]
I worked on this assignment with Chad Cummings.
\section{}
 Suppose we want to use insertion sort to sort the numbers in array
 $A[ 1\cdots n ]$. We can express insertion sort as a recursive
 procedure as follows. In order to sort $ A[1\cdots n]$, we
 recursively sort $A[1\cdots n-1]$ and then insert $A[n]$ into the
 sorted array $A[1\cdots n-1]$. Write a  recurrence for the worst-case
 running time of this recursive version of insertion sort, and solve
 the recurrence. 

\paragraph{Answer}
When we split the insertion sort into the stated recursive algorithm,
then essentially, we have the following.
\begin{equation}
  T(n) = T(n-1) + n-1
\end{equation}
This takes into account the recursive part (the $T(n-1)$ term) and
then a worst case insertion (the $n-1$ term).
Now to solve this, we can use the recursion tree method.
\begin{align*}
  k && \text{\# leaves}& & \text{Tree}&& f(n)\\
  &&&&T(n)&&\\
  1&&1&&T(n-1)&&n-1\\
  2&&1&&T(n-2)&&n-2\\
  \vdots&&\vdots&&\vdots&&\vdots\\
  n-1&&1&&T(1)&&1\\
\end{align*}
So, adding up the cost of the recursion, we get $O(1)$ from the final
leaf, and then we get 
\begin{equation}
  \sum_{i = 1}^{n-1}i = \frac{1}{2}(n-1)n.
\end{equation}
Putting these together we get
\begin{equation}
  T(n) = O( n(n-1) ) + O(1) = O(n^2).
\end{equation}


\section{}
 Let $A[1\cdots n]$ be an array of $n$ distinct numbers (i.e., no two
 numbers are equal). If $i < j$ and $A[i] > A[j]$, then the pair
 $(i,j)$ or $(A[i],A[j])$ is called an inversion of A. 
 \begin{enumerate}[label=(\alph*)]
 \item List all inversions of the array $(14,12,17,11,19)$. (5 points)
 \item[] (14, 12), (14, 11), (12, 11), (17,11)
 \item What array with elements from the set $\{1,2,...,n\}$ has the
   most inversions? How many inversions does it have? (5 points) 
 \item[]
   The array that has the most inversions is an array that is sorted
   in descending order.  The number of inversions is 
   \begin{equation}
     I(n) = \sum_{i=1}^{n-1}i = \frac{1}{2}(n-1)n.
   \end{equation}
 \item Give a divide-and-conquer algorithm that computes the number of 
   inversions in array $A$ in $O(n\log n)$ time. (Hint: Modify merge
   sort.) (20 points)
 \item[] 
   An algorithm to do the counting of the inversions can be done with
   a merge sort (with a small introduction of logic into the merge
   process).
   What needs to be added is when doing the merge, a counter is kept
   that counts the inversions.  Say that we have the left and right
   sorted arrays - along with the counted inversions in each of them
   ($c_l$ and $c_r$). An inversion will occur now if an element in the
   right array needs to be inserted into the left array.  Thus, if $m$
   is the length of the left array, and $j$ is the index of the
   pointer to the element being considered in the left array, then
   there are $m-j$ elements of the left array that are bigger than the
   current element in question.  Thus, if an element of the right
   array is less than the element of question in the left array, that
   constitutes $m-j$ inversions to add to the count.  After the merge
   is finished and this cross sum of inversions has been found, then
   the sum will be the cross inversions plue the counts from the left
   and the right ($c_l$ and $c_r$).
   This only adds constant complexity to the merge routine and thus,
   the complexity of this sort will still be that of the merge sort:
   $O(n\log n$.


%   merge(A,i,mid,k)
%   \State Create and array $B[1, \dots, k-i+1]$
%   \State $left=i$, $right=mid+1$, $mark=1$
%   \While{$left \le mid$ \And $right \le k$}
%     \If{$A[left] \le A[right]$}
%       \State $B[b] = A[a]; a++$
%     \Else
%       \State $B[b] = A[c]; c++$
%     \EndIf
%     \State $b++$
%     \Endwhile



 \end{enumerate}
Note: 0 points will be given for this problem if your algorithm is not
based on the divide-and-conquer strategy.

\section{}
 Solve the following recurrences (you may use any of the approaches we
 discussed in class). Make your bounds as small as possible (in the
 big-O notation). For each recurrence, $T(n)$ is constant for $n \le
 2$.  

 \begin{enumerate}[label=(\alph*)]
 \item $T(n) = 2\dot T\left(\frac{n}{2}\right) + n^4$.
 \item[] Answer: $T(n) = \Theta(n^4)$.

   Let's first try the Master Theorem.  We get $a = 2$, and $b = 2$,
   and thus $ \log_b(a) = 1$. Now, let's consider $f(n) = n^4$.
   Because $f(n) = \Omega(n^{1-\epsilon})$, then the first branch
   won't work.  And the second condition doesn't hold either, thus we
   need to look at the third conditions. Firstly, $f(n) =
   \Omega(n^{1+\epsilon})$, so we now need to check the second part of
   this condition.  Does there exist $c < 1$ such that $2f(\frac{n}{2})
   \le cf(n)$?
   \begin{equation}
     \begin{split}
     2f\left(\frac{n}{2}\right) &\le c f(n)\\
     2\left(\frac{n}{2}\right)^4 &\le cn^4\\
     \frac{1}{8} & \le c < 1.
     \end{split}
   \end{equation}
Let $c = \frac{1}{4}$, and the inequality is satisfied, and thus, we
can say that $T(n) = \Theta(n^4)$.
 \item $T(n) = 4\dot T\left(\frac{n}{2}\right) + n$.
 \item[] Answer: $T(n) = \Theta(n^2)$.

   Again, let's first try the Master Theorem.  We get $a = 4$, and $b
   = 2$, and thus $ \log_b(a) = 2$. Now, let's consider $f(n) = n$. 
   Well, if we let $\epsilon = \frac{1}{2}$, then $f(n) = O(n^{2 -
     \epsilon}) = O(n^{3/2})$.  Thus by the Master Theorem, we can say
   that $T(n) = \Theta(n^2)$.

 \item $T(n) = 2\dot T\left(\frac{n}{2}\right) + n\log n$.
 \item[] Answer: $T(n) = O(n\log^2 n)$.

   We will try to attack this one using the guess and verification
   method. I'm assuming that the complexity of $T(n) = O(n\log^2
   n)$. So, now we need to find a $c > 0$ such that $T(n) \le cn\log^2
   n$. We will assume that this holds true fot $n/2$. That is,
   explicitly, we are assuming that 
   \begin{equation}
     T\left(\frac{n}{2}\right) \le c
     \frac{n}{2}\log^2\left(\frac{n}{2}\right).
   \end{equation}
   
   Now we need to show that 
   \begin{equation}
     \begin{split}
       T(n) = 2T\left(\frac{n}{2}\right) + n\log n \le n\log^2\left(n\right).
     \end{split}
   \end{equation}
   Substituting our assumption into the recurrence relation we obtain
   \begin{equation}
     \begin{split}
       T(n) &=
       2\left(c\frac{n}{2}\log^2\left(\frac{n}{2}\right)\right) +
       n\log n\\
         &= cn\log^2\left(\frac{n}{2}\right) + n\log n
     \end{split}
   \end{equation}
   Ok, let's get to work
\newcommand{\q}{\overset{?}{\le}}
   \begin{equation}
     \begin{split}
       cn\log^2 \left(\frac{n}{2}\right) +n\log n
       &= cn\log \left(\frac{n}{2}\right)\log
       \left(\frac{n}{2}\right) + n\log n\\ 
       &= cn\left(\log n  - 1\right)
       \left(\log n - 1\right) + n\log n\\ 
       &= cn\left(\log^2 n  - 2\log n + 1\right) + n\log n\\  
       &= cn\log^2 n  - 2cn\log n + cn  + n\log n\\  
       &= c\left(n\log^2 n - n\log n\right) - c\left(n\log n -n\right)
       + n\log n
     \end{split}
   \end{equation}
   Ok, Now let's compare with our hypothesis.
   \begin{equation}
     \begin{split}
       c\left(n\log^2 n - n\log n\right) - c\left(n\log n -n\right)
       + n\log n &\q n\log^2 n\\
       c\left(n\log^2 n - n\log n\right) - c\left(n\log n -n\right)
       &\q n\log^2 n - n\log n. \\ 
     \end{split}
   \end{equation}
   Now, if we let $c=1$, then we obtain the following
   \begin{equation}
     \begin{split}
     \left(n\log^2 n - n\log n\right) - \left(n\log n -n\right)
       & \q n\log^2 n - n\log n \\
     - \left(n\log n -n\right)
       & \q 0\\
       -\left(\log n - 1\right) &\q 0.
     \end{split}
   \end{equation}
   The term $\log n - 1 $ is positive for $n > 1$, and thus $c=1$
   satisfies the inequality.  By this, we can also say that $T(n) =
   O(n\log^2 n)$, indeed.
 \item $T(n) = T\left(\frac{2}{3}\dot n\right) + n$.
 \item[] Answer: $T(n) = \Theta(n)$.

   For this one, We will try to Master Theorem again. 
   The constants $a$ and $b$ give us $\log_b(a) = \log_{3/2}(1) = 0$.
   Now, $f(n)$ doesn't satisfy either of the first two conditions, so
   let's find an $\epsilon > 0$ such that $f(n) = n =
   \Omega(n^{epsilon})$.  By inspection, and $\epsilon = \frac{1}{2}$
   works, thus we need to verify the second portion of the condition,
   that is, we need to find $c<1$ such that 
   \begin{equation}
     f\left(\frac{2n}{3}\right) \le cf(n).
   \end{equation}
   Thus, we get
   \begin{equation}
     \begin{split}
       \frac{2n}{3} &\le cn\\
       \frac{2}{3} & \le c < 1.
     \end{split}
   \end{equation}
   Hence we can let $c = \frac{5}{6}$, and the inequality is
   satisfied, and we can say, by the Master Theorem, that $T(n) =
   \Theta(n)$. 
 \end{enumerate}


\section{}
 Given $k$ sorted lists $L_1 ,L_2 ,\dots ,L_k$ of $n/k$ numbers each,
 with $1 \le k \le n$, design a divide-and-conquer algorithm for
 sorting all $n$ numbers in the $k$ sorted lists. Your algorithm
should run in $O(n\log k)$ (not $O(n\log n)$) time. You may assume
 the numbers in each sorted list $L_i$ , for any $1 \le i \le k$, are
 sorted in ascending order. (20 points) 

ote: 0 points will be given for this problem if your algorithm is not
based on the divide-and-conquer strategy. 

\paragraph{My answer}

The algorithm is given as
{\singlespacing
\begin{algorithmic}
\Function{mergeSortedLists}{$\{L_i\},i,j$}
  \If{$i = j$}

    \Return $L_i$;
  \EndIf
  \State $mid \gets (i+j)/2$;
  \State $L_{left} \gets $  \Call{mergeSortedLists}{$\{L_1\},i,mid$};
  \State $L_{right} \gets $ \Call{mergeSortedLists}{$\{L_1\},mid+1,j$};

  \State $L \gets $ \Call{merge}{$L_{left}, L_{right}$};

  \Return $L$;
\EndFunction
\end{algorithmic}
} % End of singlespacing
  
The algorithm merges the lists by splitting the number of lists at
each recursion step. and merging the returned lists.  The recursion
depth of this is $\log k$, and at each level, the {\sc merge} routine
(which  
is the standard/stock merge routine from the merge sort) has
$O(n)$ complexity.  Thus the entire algorithm has a complexity of 
$O(n\log k)$.  The algorithm is correct, because the list will come
out of the {\sc mergeSortedLists} merged (and sorted).




\end{document}