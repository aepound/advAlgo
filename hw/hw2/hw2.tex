\documentclass{article}
\input{commonheader}
\usepackage{comment}
\usepackage{enumitem}
\usepackage{algpseudocode}

\title{CS 5050: Homework 2}      
\author{Andrew Pound}
\date{\today}

\begin{document}
\maketitle

%\begin{enumerate}[label=(\alph*)]
I worked on this assignment with Chad Cummings.
\section{}
 Suppose we want to use insertion sort to sort the numbers in array
 $A[ 1\cdots n ]$. We can express insertion sort as a recursive
 procedure as follows. In order to sort $ A[1\cdots n]$, we
 recursively sort $A[1\cdots n-1]$ and then insert $A[n]$ into the
 sorted array $A[1\cdots n-1]$. Write a  recurrence for the worst-case
 running time of this recursive version of insertion sort, and solve
 the recurrence. 





\section{}
 Let $A[1\cdots n]$ be an array of $n$ distinct numbers (i.e., no two
 numbers are equal). If $i < j$ and $A[i] > A[j]$, then the pair
 $(i,j)$ or $(A[i],A[j])$ is called an inversion of A. 
 \begin{enumerate}[label=(\alph*)]
 \item List all inversions of the array $(14,12,17,11,19)$. (5 points)
 \item What array with elements from the set $\{1,2,...,n\}$ has the
   most inversions? How many inversions does it have? (5 points) 
 \item Give a divide-and-conquer algorithm that computes the number of 
   inversions in array $A$ in $O(n\log n)$ time. (Hint: Modify merge
   sort.) (20 points) 
 \end{enumerate}
Note: 0 points will be given for this problem if your algorithm is not
based on the divide-and-conquer strategy.

\section{}
 Solve the following recurrences (you may use any of the approaches we
 discussed in class). Make your bounds as small as possible (in the
 big-O notation). For each recurrence, $T(n)$ is constant for $n \le
 2$.  

 \begin{enumerate}[label=(\alph*)]
 \item $T(n) = 2\dot T\left(\frac{n}{2}\right) + n^4$.
 \item[]
 \item $T(n) = 4\dot T\left(\frac{n}{2}\right) + n$.
 \item[]
 \item $T(n) = 2\dot T\left(\frac{n}{2}\right) + n\log n$.
 \item[]
 \item $T(n) = T\left(\frac{2}{3}\dot n\right) + n$.
 \item[]
 \end{enumerate}


\section{}
 Given $k$ sorted lists $L_1 ,L_2 ,\dots ,L_k$ of $n/k$ numbers each,
 with $1 \le k \le n$, design a divide-and-conquer algorithm for
 sorting all $n$ numbers in the $k$ sorted lists. Your algorithm
 should run in $O(n\log k)$ (not $O(n\log n)$) time. You may assume
 the numbers in each sorted list $L_i$ , for any $1 \le i \le k$, are
 sorted in ascending order. (20 points) 

Note: 0 points will be given for this problem if your algorithm is not
based on the divide-and-conquer strategy. 





\end{document}